%%%%%%%%%%%%%%%%%%%%%%%%%%%%%%%%%%%%%%%%%%%%%%%%
% COPYRIGHT: (C) 2012-2015 FAU FabLab and others
% Bearbeitungen ab 2015-02-20 fallen unter CC-BY-SA 3.0
% Sobald alle Mitautoren zugestimmt haben, steht die komplette Datei unter CC-BY-SA 3.0. Bis dahin ist der Lizenzstatus aller alten Bestandteile ungeklärt.
%%%%%%%%%%%%%%%%%%%%%%%%%%%%%%%%%%%%%%%%%%%%%%%%


\newcommand{\basedir}{fablab-document}
\documentclass[13pt]{\basedir/fablab-document}

%\usepackage[margin=1.5cm]{geometry}
%\topmargin 0cm
%\usepackage{times} % Times New Roman oder ähnliche Schriftart
%\setkomafont{sectioning}{\normalcolor\bfseries} %auch für Überschriften
\usepackage[T1]{fontenc}
\usepackage[utf8]{inputenc}
\usepackage{wrapfig} %Textumlauf um Bilder
%\usepackage{url}
%\usepackage{hyperref}
%\hypersetup{colorlinks=true,urlcolor=blue}

%\usepackage{graphicx} % Grafiken einbinden
%\usepackage[ngerman]{babel} % benötigt Paket texlive-lang-german, wenn ein Fehler "no hyphenation patterns ..." kommt
\date{2017}
%\pagestyle{empty} % keine Seitennummern
%\sffamily
\author{kontakt@fablab.fau.de}
\title{Einweisung Biolab}
\linespread{1} % Zeilenabstand

\usepackage{parskip} % Abstände zwischen Absätzen / Listenelementen
 \setlength{\parskip}{0.7\parskip}
% \setlength{\parsep}{0pt}
% \setlength{\headsep}{0pt}
% \setlength{\topskip}{0pt}
% \setlength{\topmargin}{0pt}
% \setlength{\topsep}{0pt}
% \setlength{\partopsep}{0pt}

% \usepackage[compact]{titlesec} % Abstände bei Überschriften
% \titlespacing{\section}{0pt}{*0.5}{*0}
% \titlespacing{\subsection}{0pt}{*0.3}{*0}
%\titlespacing{\subsubsection}{0pt}{*0}{*0}

%\fancyfoot[L]{}
%\fancyfoot[C]{}
%\fancyfoot[R]{Version 2, April 2012}

\setcounter{secnumdepth}{0}

% solche Wörter werden nicht getrennt!
\hyphenation{Mechanikwerkstatt}

% Neuer Befehl \subscript (Text tiefgestellt) von http://anthony.liekens.net/index.php/LaTeX/SubscriptAndSuperscriptInTextMode
%\newcommand{\subscript}[1]{\ensuremath{_{\textrm{\small{#1}}}}}

\begin{document}
\color{red}
\hrule
\begin{center}
	\large{Achtung! Einweisung ist noch in Arbeit!}
	\vspace{0.1cm}
\end{center}
\hrule
\color{black}

\maketitle

\vbox{\vspace{1cm}}


\section{Biologische Sicherheitsstufen}
In der Biologie hat sich ein Sicherheitsstufenprinzip etabliert.
Dieses kennt Stufen von S1 bis S4.
Im FabLab sind Arbeiten die eine Schutzstufe erfordern nicht möglich.
\todo{ausführlicher beschreiben und eventuell explizite Verbote aussprechen.}

\section{Biolab-Betreuer}
Das Biolab wird betreut von Christian Schulz, Patrick Kanzler und Ulrike Schöler.

\section{Regeln}
\begin{itemize}
\item Arbeiten mit dem Biolab sind außerhalb der Workshop-Zeiten nur in Rücksprache mit den Biolab-Betreuern möglich.
\item Während dem Workshop ist das Lab für normalen Betrieb geschlossen.
\item Vor den Arbeiten ist der Arbeitsplatz zu säubern und mit einem Oberflächendesinfektionsmittel zu desinfizieren.
\item Desinfiziere genauso deine Hände mit Handdesinfektion.
\item Nach den Arbeiten wird der Arbeitsplatz gereinigt und desinfiziert.
\item Desinfiziere nach der Arbeit erneut deine Hände.
\item Stelle den ursprünglichen Zustand des Arbeitsplatzes wieder her.
\end{itemize}

\ccLicense{biolab-einweisung}{Sicherheitseinweisung Biolab}

\end{document}
